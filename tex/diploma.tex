\documentclass[14pt,a4paper]{extreport}
\usepackage[left=2cm,right=1.5cm,top=1.5cm,bottom=2.5cm,bindingoffset=0cm]{geometry}
\usepackage[T2A]{fontenc}
\usepackage[utf8]{inputenc}
\usepackage[russian]{babel}
\usepackage{cmap}
\usepackage{amsmath}
\usepackage{amsfonts}
\usepackage{amssymb} 
\usepackage{hyperref}
\usepackage{stackrel}
\usepackage{graphicx}
%\widetilde
\nonfrenchspacing
\righthyphenmin=2
\def\jo{\"e}
\def\num{\textnumero}

%%%%% Формулы
\def\le{\leqslant} % Делает знак неравенства классическим
\def\ge{\geqslant}
\def\emptyset{\varnothing} % Обычное пустое множество
\def\divisible#1#2{#1\hspace{2.5pt}\raisebox{-1.5pt}{\smash{\vdots}}\hspace{2.5pt}#2}
\def\eqmod#1#2#3{#1\equiv#2\mbox{\hspace{2mm}(mod $#3$)}}
\def\so{\Rightarrow}
\def\theoremend{\hfill\vbox to 0pt{\hbox{$\square$}}}
\def\solutionend{\hfill$\blacksquare$}
\def\quest#1{\stackrel{?}{#1}}
\def\epsilon{\varepsilon}
\def\phi{\varphi}
\def\rddots#1{\cdot^{\cdot^{\cdot^{#1}}}} % Зеркальное отражение \ddots
%\def\modop{\ \ \mbox{mod}\ \ }

\def\missing{\rule[-0.33em]{1em}{1.36em}\fbox{???}\rule[-0.33em]{1em}{1.36em}}
\def\missingwhat#1{\fbox{\parbox{10cm}{#1}}}


\def\df{{\bfseries Определение. }}
\def\pr#1{{\bf Задача \num#1. }}
\def\th{{\bfseries Теорема. }}
\def\bigmid{\mathrel{}\middle|\mathrel{}}
\def\solutionstart{$\square$\;\;}
\def\therm#1{{\it #1}}

\usepackage{indentfirst}
\DeclareMathOperator{\Mat}{Mat}
\DeclareMathOperator{\M}{M}
\DeclareMathOperator{\Sp}{Sp}
\DeclareMathOperator{\Tr}{Tr}
\DeclareMathOperator{\tr}{tr}
\DeclareMathOperator{\Def}{def}
\DeclareMathOperator{\id}{id}
\DeclareMathOperator{\odd}{odd}
\DeclareMathOperator{\Det}{Det}
\DeclareMathOperator{\Vol}{Vol}
\DeclareMathOperator{\sign}{sign}
\DeclareMathOperator{\sgn}{sgn}
\DeclareMathOperator{\Id}{Id}
\DeclareMathOperator{\ord}{ord}
\DeclareMathOperator{\adj}{adj}
\DeclareMathOperator{\rk}{rk}
\DeclareMathOperator{\re}{Re}
\DeclareMathOperator{\Ker}{Ker}
\DeclareMathOperator{\im}{Im}
\DeclareMathOperator{\Arg}{Arg}
%\DeclareMathOperator{\ch}{ch}
%\DeclareMathOperator{\ker}{ker}
\DeclareMathOperator{\Rg}{Rg}
\DeclareMathOperator{\Dom}{Dom}
\DeclareMathOperator{\diag}{diag}
\DeclareMathOperator{\Gr}{Gr}
\DeclareMathOperator{\Tor}{Tor}
\DeclareMathOperator{\ort}{ort}
\DeclareMathOperator{\PR}{pr}
\DeclareMathOperator{\grad}{grad}
\DeclareMathOperator{\rank}{rank}
\DeclareMathOperator{\Char}{char}
\DeclareMathOperator{\Int}{int}
\DeclareMathOperator{\Ext}{ext}
\DeclareMathOperator{\diam}{diam}
\DeclareMathOperator{\cov}{cov}
\DeclareMathOperator{\dom}{dom}

\def\deflr{\stackrel[\Def]{}{\Longleftrightarrow}}

\def\dvec#1#2{\begin{pmatrix}
#1\\#2
\end{pmatrix}}
\def\ov#1{\overrightarrow{#1}}
\def\ovl#1{\overline{#1}}
\def\mat#1#2#3{\begin{pmatrix}
#1_{11} & #1_{12} & \ldots & #1_{1#3}\\
\vdots &  & \ddots & \vdots \\
#1_{#2 1} & #1_{#2 2} & \ldots & #1_{#2#3}
\end{pmatrix}
}
\def\vec#1#2{\left(\begin{smallmatrix}#1\\\vdots\\
#2\end{smallmatrix}\right)}
\def\system#1#2{\left\{
\begin{aligned}
#1\\
\vdots\\
#2\\
\end{aligned}
\right.
}
\def\blockmatrix#1#2#3#4{\left(
\begin{array}{c|c}
  #1 & #2 \\ \hline
  #3 & #4
\end{array}
\right)}


\def\Uc{\stackrel{\circ}{U}}
\def\rra#1{\stackrel{#1}{\rightrightarrows}}
\def\E{\mathbb E}
\def\D{D}
\def\vrt#1{\left\lVert#1\right\rVert}
\def\br#1{\left(#1\right)}
\def\brc#1{\left\{#1\right\}}
\def\abs#1{\left|#1\right|}
% \stackrel stackbin 

\begin{document}

\chapter{Алгоритмы визуализации данных на клиенте}

В данной главе описываются реализации двух алгоритмов визуализации данных: Fruchterman Reingold и t-SNE.

Несмотря на наличие множества реализаций этих алгоритмов на JavaScript, ни одна из них не использует GPU. При этом нет простого способа портировать существующие реализации этих алгоритмов на GPU с других языков программирования, так как WebGL API очень ограничен по сравнению с OpenGL и другими API для графических карт.

\section{Описание API WebGL для обработки данных}



\section{Fruchterman Reingold}

Это классический алгоритм рисования графа из категории force directed drawing.

Он использует физическую аналогию, в которой ребрам соответствуют пружины, а вершинам --- одинаково заряженные частицы. Таким образом, соединенные ребром вершины стремятся быть на фиксированном расстоянии друг от друга, равном оптимальной длине пружины, а не соединенные вершины отталкиваются друг от друга.

На практике, закон Гука для силы пружины не используется, так как он слишком сильно действует на соединенные вершины, находящиеся в разных частях графа.

Для алгоритма Fruchterman Reingold силы притяжения и отталкивания равны $f_a(d) = d^2 / k$ и $f_r(d) = -k^2 / d$, где $k$ --- оптимальная длина пружины, при которой эти силы сбалансированы.

Физическая симуляция осуществляется при помощи интегрирования Верле. Этот метод позволяет перейти от предыдущего и текущего положения к следущему, зная только ускорение (то есть сумму сил, действующих на частицу) и не вычисляя скорости.

Алгоритм начинает со случайных позиций вершин и останавливается, когда система достигает равновесия. Для ускорения этого процесса вводится понятие {\itshape температуры}, которая ограничивает максимальное изменение положения вершин и уменьшается по заданному закону.

\subsection{Архитектура вычисления}

\begin{verbatim}
(edges, positions) -> forces_a
(positions) -> forces_r
\end{verbatim}

\subsection{Вычисление сил притяжения}

Силы притяжения.

\section{Barnes-hut}

\end{document}
